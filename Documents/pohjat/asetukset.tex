%
% T�ss� esimerkkipohjassa oletetaan, ett� projektiryhm�n nimi on esim.
% Muuta tiedostoa vastaamaan projektisi tietoja.
%

% Times-pakkaus on hy�dyllinen, koska sen avulla tulee siist� PDF:��.
\usepackage{times}

% Seuraavaan pit�� muuttaa projektiryhm�n nimi.
\defprojectgroup{Kohahdus}

% Ryhm�n j�senet. Viimeisen j�senen j�lkeen ei tule \\-merkint��
\defgroupmembers{
  \membername{Taro Morimoto, Projektip��llikk�}\\
  \membername{Tuomas Palmanto, Vaatimusm��rittelyvastaava}\\
  \membername{Mikko Kinnunen, Suunnitteluvastaava}\\
  \membername{Markus Kivil�, Koodivastaava}\\
  \membername{Jari Inkinen, Testausvastaava}\\
  \membername{Paula Kuosmanen, Dokumenttivastaava}
}

% Projektin asiakkaan nimi. Jos asiakkaita on useita, nimet��n heist� yksi
% ensisijaiseksi asiakaskontaktiksi.
\defprojectclient{Teemu Kerola}

% Projektiryhm�n kotisivu.
\defprojecthomepage{http://www.cs.helsinki.fi/group/kohahdus}

% Projektin johtoryhm� eli vastuuhenkil�t.
\defprojectmasters{
  \mastername{Sanna Keskioja}
}

% Huomautukset
\def\defnote{ Huom! T�m� on esimerkkiryhm�.  T�llaista ryhm�� ei ole
koskaan ollut. Kaikki yhteydet todellisiin henkil�ihin ovat sattumaa.
Kaikki dokumentit ovat Juha Tainan kirjoittamia 2005-2006. }

%
% Omat, kaikille projektin dokumenteille yhteiset asetukset voi laittaa
% t�nne.
%
