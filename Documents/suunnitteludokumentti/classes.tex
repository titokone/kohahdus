\subsection{Criterion}
public abstract class Criterion

{\bf Base class for all criterion types. Each task has a fixed number of criteria
for analyzing the student's solution. Criterion may be used to pass/fail the solution,
evaluate the quality of the solution, or do both/neither.

Different types of criteria are implemented as Criterion-subclasses. Task composer used
for creating new tasks uses mostly the interfaces defined here, but also may use some
methods specifc to the subclass. Task analyzer component used for checking the student's
solution uses only this abstract interface. The analyzer is not aware of the details
of the subclasses.

This class and all sub-classes shall provide following quarantees:
\begin{itemize}
\item getters always return Strings, regardless of the field type
\item getters never return null, but they may return empty strings
\item setters always take Strings, regarless of the field type. Invalid strings, (eg. non-numeric
      string for numeric field), empty strings, and null values are acceptable and will either clear
      the field or set the field to some default value.
\item neither getter nor setters will ever throw exceptions
\item Criterion objects are never in an invalid state. This is done by setting and validating
      mandatory fields in the constructor. However, Criterion object deserialized from the database
      are not subject to validation, they are assumed to be always valid.
\end{itemize}

\begin{verbatim}
Criterion (abstract)
+- InstructionCriterion (abstract)
|  +- ForbiddenInstructionsCriterion
|  +- RequiredInstructionsCriterion
+- MeasuredCriterion (abstract)
|  +- CodeSizeCriterion
|  +- DataAreaSizeCriterion
|  +- DataReferencesCriterion
|  +- ExecutetionStepsCriterion
|  +- MemReferencesCriterion
|  +- StackSizeCriterion
+- ScreenOutputCriterion
+- VariableCriterion (abstract)
   +- RegisterCriterion
   +- SymbolCriterion
\end{verbatim}

For sake of brevity, Criterion subclasses are not described in this document.}

{\tt	protected static final long UNDEFINED = Long.MIN\_VALUE;}\\
Special case for signaling undefined numeric value. Criteria that deal with numeric
types (such as RegisterCriterium) need all 32-bits of integer, but we also need a way
to represent undefined (ie. null) values. Best-but-still-ugly solution is to use
longs for all numeric types and use a special signal value for null.

{\tt	protected Criterion()}\\
Empty constructor for deserialization\\	

{\tt	protected Criterion(String id, boolean usesSecretInput)}\\
Constructor to initialize mandatory data members of Criterion.\\
{\tt@param id} Identifier that can used to distinguish the different criteria of one task\\
{\tt@param usesSecretInput} true if criterion is to be used in conjunction to secret input\\

{\tt	public String getId()}\\
Return the identifier of this Criterion\\

{\tt	public abstract String getName(ResourceBundle languageBundle)}\\
Return human-readable name of this criterion\\
{\tt@param languageBundle} Bundle containing language-dependant parts of crierion names\\

{\tt 	public boolean isSecretInputCriterion()} \\
Return true if this criterion is to be used with secret input \\

{\tt 	public String getFailureFeedback()} \\
Return feedback string used solution attempts that do not pass the acceptance test\\

{\tt 	public String getAcceptanceFeedback()} \\
Return feedback string used for solution attempts that pass the acceptance test but not the quality test\\

{\tt	public String getHighQualityFeedback()} \\
Return feedback string used for for solution attempts that pass pass the acceptance test, and also
the quality test if one is defined.\\

{\tt 	public String setFailureFeedback()} \\
Set failure feedback\\

{\tt 	public String setAcceptanceFeedback()} \\
Set acceptance feedback\\

{\tt	public String setHighQualityFeedback()} \\
Set high quality feedback\\

{\tt 	public abstract boolean hasAcceptanceTest(boolean usingModelAnswer)}\\
Return true if this criterion has test for evaluating failure/success of the
student's answer. Return of false means this criterion should NOT be used to
test passesAcceptanceTest(..)\\
{\tt@param usingModelAnswer} True if the inspection method for this task is model-answer\\

{\tt	public abstract boolean passesAcceptanceTest(TitoState studentAnswer, TitoState modelAnswer)} \\
Return true if student's solution meets the passing requirement of this Criterion. The behaviour
of this method is undefined if called despite a false return from hasAcceptanceTest(..)\\
{\tt@param studentAnswer} end state of TitoKone for student's answer\\
{\tt@param modelAnswer} end state of TitoKone for teacher's answer, or null\\

{\tt	public abstract String getAcceptanceTestValue()}\\
Return the value the student's answer will be compared to\\

{\tt	public abstract void setAcceptanceTestValue(String test)}\\
Set the value the student's answer will be compared to\\

{\tt 	public boolean hasQualityTest(boolean usingModelAnswer)}\\
Return true if this criterion has test for evaluating quality of the student's answer.
Criterion class provides a default implementation that always returns false\\

{\tt	public boolean passesQualityTest(TitoState studentAnswer, TitoState modelAnswer)} \\
Return true if student's solution meets the quality requirement of this Criterion.
Criterion class provides a default implementation that always returns false\\

{\tt	public String getQualityTestValue()}\\
Return the value the student's answer will be compared to.
Criterion class provides a default implementation that always returns an empty string\\

{\tt	public abstract void setQualityTestValue(String test)}\\
Set the value the student's answer will be compared to.
Criterion class provides a default implementation that is a no-op\\

{\tt 	public String serializeToXML()} \\
Return a serialized copy of this Criterion in XML-format\\

{\tt 	protected abstract String serializeSubClass();} \\
Serialize non-static data-members of Criterion sub-class to XML format. The subclass
can freely decide the names of the XML tags. The abstract Criterion class will handle
the serialization of its data-members. The serialized string is stored in the eAssari
database in a 2000-char field so subclasses should try to keep the tags and data short
(without being cryptic). \\
	
{\tt 	protected abstract void initSubClass(String serializedXML);} \\
Initialize non-static data-members of this Criterion subclass instance using the 
serialized representation returned by serializeToXML. Data-members of the
abstract Criterion class will have already been deserialized when this method
is called. \\
	
{\tt 	public static Criterion deserializeFromXML(String xml)} \\
Instantiate new Criterion object using the serialized form Xml \\
	
{\tt 	protected static String toXML(String tagname, boolean value)} \\
Serialize boolean value to XML string. Helper function for serializeSubClass() \\
	
{\tt 	protected static String toXML(String tagname, String value)} \\
Serialize String value to XML string. Helper function for serializeSubClass() \\
	
{\tt 	protected static String toXML(String tagname, long value)} \\
Serialize integer value to XML string. Helper function for serializeSubClass() \\

{\tt 	protected static boolean parseXMLBoolean(String XML, String tagname)} \\
Deserialize boolean value from XML string. Helper function for initSubClass() \\

{\tt 	protected static String parseXMLString(String XML, String tagname)} \\
Deserialize String value from XML string. Helper function for initSubClass() \\
	
{\tt 	protected static int parseXMLLong(String XML, String tagname)} \\
Deserialize integer value from XML string. Helper function for initSubClass() \\














\subsection{DBHandler}
public class DBHandler

{\bf Singleton class used for database interactions. Each public method of DBHandler class
encapsulates one database transaction, and thus may cause multiple inserts/updates/removes
with one call. The atomicity of the operations is quaranteed by using the transaction model
provided by the SQL standard.} \\
    
{\tt	private DBHandler()} \\
Prevent outside instatiation by giving constructor private scope\\

{\tt	public static DBHandler getInstance()} \\
Return DBHandler instance\\

{\tt	private boolean init()} \\
Initialize db connection pool\\

{\tt	protected void release(Connection conn) throws SQLException} \\
Closes db connection\\

{\tt	protected Connection getConnection () throws SQLException} \\
Load database driver if not already loaded\\

{\tt	public LinkedList<Course> getCourses() throws SQLException} \\
Returns all courses\\

{\tt	public synchronized boolean createCourse(Course course) throws SQLException} \\
Adds new course to database\\

{\tt	private boolean addCourse(Course course) throws SQLException} \\
Add new course to database. Does not check whether the course already exists in the DB.\\

{\tt	public boolean removeCourse(String courseID) throws SQLException} \\
Remove course from database (and thus all refering taskinmodules).\\

{\tt	private boolean addModule(String courseID) throws SQLException} \\
Add a default module to course.\\

{\tt	private boolean addTaskInModule(String courseID, String taskID) throws SQLException} \\
Adds a taskinmodule entry so that the task is linked with the course.\\

{\tt	public LinkedList<Task> getTasks() throws SQLException} \\
Return all tasks\\
	
{\tt	public Task getTask(String taskID)} \\
Return task identified by taskID\\

{\tt	public synchronized boolean createTask(Task task, List<Criterion> criteria) throws SQLException} \\
Add new task to task database. The insert will affect all courses. This operation
will also create the criteria for the task\\

{\tt	public boolean addTask(Task task) throws SQLException} \\
Adds a task to DB without any linkage to courses and modules\\

{\tt	public boolean updateTask(Task task, List<Criterion> criteria) throws SQLException} \\
Update existing task. The update will affect all courses. This operation
will also update the criteria of the task.\\

{\tt	private boolean updateTask(Task task) throws SQLException}\\
Updates a task to DB without modifying any courses or modules\\

{\tt	public boolean removeTask(Task task)} \\
Remove task from task database (and thus all courses). This will also remove all stored
answers of the task.\\

{\tt 		private boolean addCriterion(Task t, Criterion c)} \\
Add criterion c to task. Helper for addTask(..) and updateTask(..)\\

{\tt	private boolean updateCriterion(Task t, Criterion c) throws SQLException}\\
Updates criterion c of task\\

{\tt 		private void removeCriteria(Task t)} \\
Remove criteria form task. Helper for removeTask(..) and updateTask(..)\\

{\tt	public LinkedList<Criterion> getCriteria(Task task) throws SQLException}\\
Return the criteria of task\\

{\tt	public CriterionMap getCriteriaMap(Task task) throws SQLException}\\
Return the criteria of a task in a map\\

{\tt	public User[] getUsers(Course course) throws SQLException} \\
Return all users who have attempted to solve at least one task of Course\\

{\tt	public User getUser(String userID, String password) throws SQLException} \\
Return user identified by userID and password (login)\\

{\tt	public User getUser(String userID) throws SQLException} \\
Return user identified by userID\\

{\tt	public boolean createUser(User user)} \\
Add new user to user database. Does not check weather the user already exists in the DB.\\
	
{\tt	public boolean updateUser(User user)} \\
Update existing user to DB with userID. Does not check weather the user exists or not.\\

{\tt 	public void removeUser(String userID)} \\
Removes all student's answers and records from database (studentmodel, storedanswer and eauser).\\

{\tt	public String[][] getAnswerStatistics(String courseID)} \\
Returns a two dimensional matrix of stored answers (table studentmodel) 
of tasks and student names.\\

{\tt	public User[] searchStudent(String name)} \\
Returns a list of students (Users) that matches the param name 
with first or lastname in the eauser table.\\

{\tt	public LinkedList getStudentAnswers(String userID)} \\
Returns a list of tasks that student has answered. Each task on the list is 
a hashtable with columns as entries.\\

{\tt	public String storeStateAndAnswer(boolean shouldStore, User student, String courseId, 
									   String taskID, String[] answers, boolean accepted,
									   String lang, Feedback feedback) throws SQLException } \\
Stores the answer of the student into database table storedanswer. 
Also stores the state of the student's answered task into table studentmodel.\\





\subsection{Task}
public class Task

{\bf Modify existing Task class by adding the following methods}\\

{\tt	public Task()} \\
Creates a new instance of Task\\

{\tt	public Task(String taskID)} \\
Creates a new instance of Task\\

{\tt	public setTaskID(String taskID)} \\
Sets ID of task\\

{\tt	public setLanguage(String lang)} \\
Set the preferred language of this user. The language is either "EN" or "FI\\

{\tt	public void setName(String name)} \\
Set the name of this task\\

{\tt	public void setAuthor(String name)} \\
Set "last task modification by" attribute to Name, set last-modification-timestamp to current date and time\\

{\tt	public void setDescription(String desc)} \\
Set description (teht�v�nanto) of this task\\

{\tt	public void setPublicInput(String input)} \\
Set the public input as String of this task\\

{\tt	public void setSecretInput(String input)} \\
Set the secret input as String of this task\\

{\tt	public void setModelAnswer(String code)} \\
Set model answer code\\

{\tt	public void setCategory(String categ)} \\
Set task category of this task\\

{\tt	public void setTitoTaskType(String type)} \\
Set task type of this task\\

{\tt	public void setFillInPreCode(String code)} \\
Set the code that is prepended before student's code in a fill-in task\\

{\tt	public void setFillInPostCode(String code)} \\
Set the code that is appended to student's code in a fill-in task\\

{\tt	public void setFillInTask(boolean fillIn)} \\
Set this task as fill-in or create-full-program\\

{\tt	public void setNoOfTries(int noOfTries)} \\
Sets number of tries for student\\

{\tt	public void setValidateByModel(boolean useModel)} \\
Set the validation method of this task\\

{\tt	public void setModificationDate()} \\
Set the date and time this task was last modified\\

{\tt	public void setHasSucceeded(boolean hasSucceeded)} \\
Set the hasSucceeded true or false depending the student has passed the task\\

{\tt	public void setPassFeedBack(String pass)} \\
Set return feedback if student passes the task\\

{\tt	public void setFailFeedBack(String fail)} \\
Set return feedback if student fails the task\\

{\tt	public void setMaximumNumberOfInstructions(int num)} \\
Endless loop prevention\\

{\tt	public void setMaximumNumberOfInstructions(String num)} \\
Endless loop prevention\\

{\tt	public String getTaskID()} \\
Return id of this task\\

{\tt	public String getLanguage()} \\
Return the language of this task\\

{\tt	public String getName()} \\
Return the name of this task\\

{\tt	public String getAuthor()} \\
Return name of the last person who has modified this task\\

{\tt	public String getDescription()} \\
Return the description (teht�v�nanto) of this task\\

{\tt	public String getPublicInput()} \\
Return the public input as String of this task\\

{\tt	public String getSecretInput()} \\
Return the secret input as String of this task\\

{\tt	public String getModelAnswer()} \\
Return code of the model answer provided by teacher\\

{\tt	public String getCategory()} \\
Return the task category of this task\\

{\tt	public String getTitoTaskType()} \\
Return the task type of this task\\

{\tt	public String getFillInPreCode()} \\
Return the code that is prepended before student's code in a fill-in task\\

{\tt	public String getFillInPostCode()} \\
Return the code that is appended to student's code in a fill-in task\\

{\tt	public int getNoOfTries()} \\
Return the number of tries student has tried to make this task\\

{\tt	public boolean isFillInTask()} \\
Return true if this is a fill-in task\\

{\tt	public boolean isProgrammingTask()} \\
Return true if this is a programming task\\

{\tt	public boolean isValidateByModel()} \\
Return true if this task is to be validated by comparing results of the student's
answer to results of teacher's answer\\

{\tt	public Date getModificationDate()} \\
Return the date and time this task was last modified\\

{\tt	public boolean isHasSucceeded()} \\
Return true if student has made the task, false if she/he hasn't\\

{\tt	public String getPassFeedBack()} \\
Return feedback if student passes the task\\

{\tt	public String getName()} \\
Return the name of this task\\

{\tt	public String getFailFeedBack()} \\
Return feedback if student fails the task\\

{\tt	public int getMaximumNumberOfInstructions()} \\
Returns loop prevention value\\

{\tt	public String getDateAsString()} \\
Returns String of the date and time\\

{\tt	protected static String parseXMLString(String XML, String tagname)} \\
Deserialize String value from XML string. Helper function for initSubClass()\\

{\tt	protected static boolean parseXMLBoolean(String XML, String tagname)} \\
Deserialize boolean value from XML string. Helper function for initSubClass()\\

{\tt	protected static int parseXMLint(String XML, String tagname)} \\
Deserialize int value from XML string. Helper function for initSubClass() return value or 0\\

{\tt	public int getMaximumNumberOfInstructions()} \\
Returns loop prevention value\\

{\tt	protected static String toXML(String tagname, int value)} \\
Serialize long value to XML string. Helper function for serializeSubClass()\\

{\tt	protected static String toXML(String tagname, String value)} \\
Serialize String value to XML string. Helper function for serializeSubClass()\\

{\tt	public String serializeToXML()} \\
Return a serialized this Task class into XML-format\\

{\tt	public void deserializeFromXML(String xml)} \\
Instantiate the Task class's parameters using the serialized form XML. 
@throws RuntimeException exceptions in the deserialization are 
caught and rethrown as uncheckced exception.\\

{\tt	public Date getModificationDate()} \\
Return the date and time this task was last modified\\

{\tt	public String toString()} \\
Returns String from the task\\





\subsection{User}
public class User 

{\bf Modify existing User class by adding the following methods}\\

{\tt	public static final String STATUS\_STUDENT = "student";} \\ 
{\tt	public static final String STATUS\_TEACHER = "teacher";} \\ 
{\tt	public static final String STATUS\_ADMIN = "adm";} \\ 
	
{\tt	public User()} \\ 
Construct unitialized User object\\
	
{\tt	public User(String userID)} \\ 
Construct unitialized User object with userid\\


{\tt	public void setUserID(String userID)} \\ 
Set userID of this user\\
	
{\tt	public void setLastName(String lastname)} \\ 
Set last name of this user\\
	
{\tt	public void setFirstName(String firstname)} \\ 
Set first name of this user\\
	
{\tt	public void setEmail(String email)} \\ 
Set email address of this user\\

{\tt	public void setStatus(String status)} \\  
{\tt	throws IllegalArgumentException if Status is not a valid Status string}\\
Set user status (teacher / student)\\

{\tt	public void setStudentNumber(String studentnum)} \\ 
Set student number of of this user\\

{\tt	public void setSocialSecurityNumber(String ssn)} \\ 
Set social security number of this user\\

{\tt	public void setPassword(String pass)} \\ 
Set password of this user to Pass\\

{\tt	public void setLanguage(String lang)} \\ 
Set the preferred language of this user. The language is either "EN" or "FI\\

{\tt	public String getUserID()} \\ 
Return userID of this user\\

{\tt	public String getLastName()} \\ 
Return last name of this user\\

{\tt	public String getFirstName()} \\ 
Return first name of this user\\

{\tt	public String getEmail()} \\ 
Return email address of this user\\

{\tt		public String getStatus()} \\
Return status of this user\\

{\tt	public String getStudentNumber()} \\ 
Return student number of this user. This identifier maps to <code>aeuser.extid</code> in the database\\

{\tt	public String getSocialSecurityNumber()} \\ 
Return social security number of this user. This identifier maps to <code>aeuser.extid2</code> in the database\\

{\tt		public String getPassword()} \\
Return plaintext password of this user\\

{\tt		public String getLpref()} \\
Return the preferred language of this user as String. The language is either "EN" or "FI"\\


{\tt	public boolean isTeacher()} \\ 
Return true of this user has the privelidges to add/remove/modify tasks and browse user statistics\\

{\tt	public boolean isStudent()} \\ 
Return true of this user has student privelidges\\

{\tt	public boolean isAdmin()} \\ 
Return true of this user has admin privelidges \\

{\tt	public boolean isValid()} \\
Test validity of this user object. The object is considered valid if all data members are set with non-empty values.\\

{\tt	private boolean isEmptyString(String str)} \\
Return false if str is null or empty "" string\\





\subsection{Course}
public class Course

{\bf Simple class for holding basic course information}\\

{\tt	public Course(String name, String id)} \\
Create new Course instance using the specified name and ID\\
	
{\tt    public String getName()} \\
Return name of this course\\

{\tt    public void setName(String name)} \\
Set name of this course\\

{\tt    public String getID()} \\
Return course ID of this course\\

{\tt    public String toString()} \\
Return name and id of this course as one String\\





\subsection{TitoAnalyzer}
public class TitoAnalyzer

{\bf Class for running TitoKone and analyzing student's answer}\\

{\tt    public TitoFeedback Analyze(Task task, List<Criterion> criteria, String programCode, String keyboardInput)} \\
Analyzes student's answercode. Returns feedback from analysis.\\




\subsection{TitoFeedback}
public interface TitoFeedback

{\bf Interface for feedback data}\\

{\tt    public TitoState getTitoState()} \\
Return the end-state of TitoKone run\\

{\tt    public String getOverallFeedback()} \\
Return the overall task feedback\\

{\tt    public String getCompileError()} \\
Return compile-time error message (null if no errors)\\

{\tt    public String getRunError()} \\
Return run-time error message (null if no errors)\\

{\tt    public boolean wasSuccessful()} \\
Return true if program was compiled and executed successfully, and all acceptance crietia were met\\

{\tt    public java.util.List<TitoCriterionFeedback> getCriteriaFeedback()} \\
Return a list criteria feedbacks\\




\subsection{TitoCriterionFeedback}
public class TitoCriterionFeedback

{\bf Class for criterion feedback information}\\

{\tt    public TitoCriterionFeedback(String name, String feedback, boolean success)} \\
Creates a new instance of TitoCriterionFeedback.\\

{\tt    public String getName()} \\
Return criterion name\\

{\tt    public String getFeedback()} \\
Return feedback of this criterion\\

{\tt    public boolean passedAcceptanceTest()} \\
Return true if this criterion meets passing requirements\\





\subsection{TitoState}
public class TitoState

{\bf Capsulates the end-state of single run of TitoKone and hides 
implementation details of TitoKone behind a single class interface.

This class will provide methods for exuting TTK91 programs and 
querying the resulting TitoKone end state}\\

{\tt	public String compile(String sourceCode)}\\
Compiles given source code. Returns compile time error message, or null if compilation succeeded 
withour errors. If this method executes successfully (returns null), the program may be run with
the run() method.\\
@param sourceCode TKK91 program source code as single string\\

{\tt	public String execute(String keyboardInput, int maxExecutionSteps)}\\
Runs previously compiled program. Returns runtime error message, or null if the program
ran to completion without errors.\\
@param keyboardInput keyboard input for the program, as [ $\backslash$n$\backslash$r$\backslash$t$\backslash$f,.:;] separated list of numbers\\
@param maxExecutionSteps maximum number instruction to execute (prevent inifinite loops)\\

{\tt	public int getRegister(int registerCode)} \\
Return contents of register\\
@param registerCode one of REG\_* constants defined in fi.hu.cs.ttk91.TTK91Cpu*\\


{\tt	public int getMemoryLocation(int address)} \\
Return contents of specified memory address\\
	
{\tt	public Map getSymbolTable()} \\
Return symbol table that maps symbol names to symbol value addresses\\
	
{\tt	public String getScreenOutput()} \\
Return TitoKone output as String in format "1234, 1234, 1234". The returned string may be empty, but it is never null.\\

{\tt	public int getStackMaxSize()} \\
Return maximum stack size reached during program execution\\

{\tt	public int getExecutionSteps()} \\
Return number of executed instructions\\
	
{\tt	public int getCodeSize()} \\
Return number of instruction words in the program code\\

{\tt	public int getDataSize()} \\
Return number of words in program's data-area\\
	
{\tt	public int getMemoryAccessCount()} \\
Return number memory references made during program run. This number includes references caused by both data and instruction fetches\\
	
{\tt	public Set<String> getUsedOpcodes()} \\
Return used opcodes in Set of Strings\\

\subsection{LanguageManager}
public class LanguageManager

{\tt	private static HashMap<String, ResourceBundle> bundles } \\
Stores all initialized ResourceBundles. \\	
	 
{\tt	public static synchronized void loadTextResources(String propertiesFile) }\\
Initializes all resources specified in parameter file, which contains filenames of 
all resources to be loaded in XML format. \\
	
{\tt	public static ResourceBundle getTextResource(String lang, String page) }\\
Returns a ResourceBundle object containing all language specific data 
on a single JSP page. \\



\subsection{TitoBundle}
public class TitoBundle extends ResourceBundle 

{\tt	private Properties xmlData} \\
Stores data from xml-file \\

{\tt	private String language} \\
Language of returned data \\
	

{\tt	public TitoBundle(Properties data, String lang) }\\	
Initalize a new TitoBundle, which returns data from Properties object in a given language \\

{\tt	protected Object handleGetObject(String key)}\\
Returns an object, which matches the key, from xmlData. It's preferable to 
use getString method from parent class instead \\

{\tt	public Enumeration<String> getKeys()} \\
Returns all available key values \\



